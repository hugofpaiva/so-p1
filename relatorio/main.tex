\documentclass[10pt,portuguese]{article}

\usepackage{fourier}

\usepackage[]{graphicx}
\usepackage[]{color}
\usepackage{xcolor}
\usepackage{alltt}
\usepackage{listings}
\usepackage[T1]{fontenc}
\usepackage[utf8]{inputenc}
\setlength{\parskip}{\smallskipamount}
\setlength{\parindent}{0pt}
\usepackage{listings}
\usepackage{setspace}
\usepackage{hyperref}


% Set page margins
\usepackage[top=100pt,bottom=100pt,left=68pt,right=66pt]{geometry}

% Package used for placeholder text
\usepackage{lipsum}

% Prevents LaTeX from filling out a page to the bottom
\raggedbottom


\usepackage{fancyhdr}
\fancyhf{} 
\fancyfoot[C]{\thepage}
\renewcommand{\headrulewidth}{0pt} 
\pagestyle{fancy}

\usepackage{titlesec}
\titleformat{\chapter}
   {\normalfont\LARGE\bfseries}{\thechapter.}{1em}{}
\titlespacing{\chapter}{0pt}{50pt}{2\baselineskip}

\usepackage{float}
\floatstyle{plaintop}
\restylefloat{table}

\usepackage[tableposition=top]{caption}



\frontmatter

\definecolor{light-gray}{gray}{0.95}

\begin{document}

\selectlanguage{portuguese}

\begin{titlepage}
	\clearpage\thispagestyle{empty}
	\centering
	\vspace{2cm}

	
	{\Large  Sistemas Operativos \par}
	\vspace{0.5cm}
	{\small Professor: \\
	José Nuno Panelas Nunes Lau\par}
	\vspace{4cm}
	{\Huge \textbf{Estatísticas de utilizadores em bash}} \\
	\vspace{1cm}
	\vspace{4cm}
	{\normalsize Carolina Araújo, 93248 \\ 
	             Hugo Paiva, 93195
	   \par}
	\vspace{2cm}

    \includegraphics[scale=0.20]{images/logo_ua.png}
    
    \vspace{2cm}
    
	{\normalsize DETI \\ 
		Universidade de Aveiro \par}
		
	{\normalsize 29-11-2019 \par}
	\vspace{2cm}
	
	\pagebreak

\end{titlepage}
\tableofcontents{}
\clearpage

\section{Introdução}
Este trabalho prático foi baseado no desenvolvimento de scripts em bash que permitem recolher algumas estatísticas sobre o modo como os utilizadores estão a usar o sistema computacional. 
\\\\
Estas ferramentas permitem visualizar o número de sessões e o tempo total de ligação para uma selecção de utilizadores e um determinado período de tempo, permitindo também a comparação dos dados obtidos em períodos distintos.
\\\\
Para desenvolver estas ferramentas com os resultados expectáveis é necessário compreender o funcionamento da Bash. Sendo mantida pelo famoso projeto GNU, a Bash, é uma ferramenta extremamente eficiente encontrada na maioria dos sistemas baseados em UNIX. É também altamente personalizável e, por isso, muito usada no mundo da programação.
 

\clearpage

\section{Preparação}
Antes de avançar com qualquer desenvolvimento em código, procedeu-se à cópia do ficheiro "/var/log/wtmp" do computador do grupo. Este ficheiro contém o histórios de todos os logins e logouts associados ao computador em questão, fornecendo todos os dados necessários para o desenvolvimento dos scripts. Desta forma, foi permitida a sua leitura sem necessidade de se estar conectado ao computador nas instalações do DETI. 
\clearpage

\section{Análise da Implementação}
\subsection{...}
\clearpage

\section{Bibliografia}

\bibliographystyle{plain}

\bibliography{biblist}

\vspace{5mm} %5mm vertical space

[1] \url{https://pplware.sapo.pt/linux/personalize-a-prompt-de-comandos-da-bash-no-linux/}


\end{document}

